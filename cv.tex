\documentclass[11pt,a4paper,sans]{moderncv}

%% ModernCV themes
\moderncvstyle{classic}
\moderncvcolor{green}
\renewcommand{\familydefault}{\sfdefault}
\setlength{\makecvtitlenamewidth}{14cm}
\nopagenumbers{}

%% Character encoding
\usepackage[utf8]{inputenc}

%% Adjust the page margins
\usepackage[scale=0.78]{geometry}

%% Personal data
\firstname{Michal}
\familyname{Zimmermann}
\address{Ovesná 23}{77900 Olomouc}
\mobile{+420~603~513~860}
\email{zimmicz@gmail.com}
\social[linkedin][cz.linkedin.com/pub/michal-zimmermann/29/8/b30/]{Michal Zimmermann}
\social[twitter]{zimmicz}
\homepage{www.zimmi.cz}
\extrainfo{narozen 1989}
%\photo[100pt][1pt]{profil.jpeg}
%\quote{Some quote (optional)}

%%------------------------------------------------------------------------------
%% Content
%%------------------------------------------------------------------------------
\begin{document}
\makecvtitle


\section{Zkušenosti}

\cventry{2012--}{Vedoucí oddělení geoinformatiky}{ENVIPARTNER, s.r.o.}{Brno}{}{GIS a IT zajištění tvorby digitálních povodňových plánů}{}
\cvitem{Náplň:}{správa Geoserveru, databáze PostGIS, rozvoj mapových aplikací, automatizace tvorby digitálních povodňových plánů}

\cventry{2013}{Spoluautor atlasu}{Masarykova univerzita}{Brno}{}{tvorba map se zdravotnickou tematikou za pomoci experimentálních kartografických metod}{}
\cvitem{Projekt:}{Experimentální atlas vybraných civilizačních chorob}

\cventry{2013}{Tvůrce mapy}{Masarykova univerzita}{Brno}{}{vytvoření interaktivních slepých map Indie a Číny pro účely výuky za použití Leaflet JS}{}
\cvitem{Projekt:}{Slepé mapy Indie a Číny}

\cventry{2012}{GIS operátor}{Ekotoxa, s.r.o.}{Olomouc}{}{identifikace rybníků na mapách vojenského mapování a stabilního katastru a následný zákres}{}
\cvitem{Projekt:}{Zákres rybniční sítě ze starých map ČR}

\cventry{2010}{GIS operátor}{Ekotoxa, s.r.o.}{Olomouc}{}{identifikace erozí ohrožených údolnic, jejich klasifikace a zákres}{}
\cvitem{Projekt:}{Zákres nestabilizovaných drah soustředěného odtoku na orné půdě v rámci ČR}


\section{Vzdělání}
\cventry{2012--2014}{Magistr}{Masarykova Univerzita}{Brno}{Geografická kartografie a geoinformatika}{}  % arguments 3 to 6 can be left empty
\cvitem{Téma práce:}{\emph{Možnosti dolování a vizualizace dat ze sociálních sítí}}

\cventry{2009--2012}{Bakalář}{Masarykova Univerzita}{Brno}{Geografická kartografie a geoinformatika}{}  % arguments 3 to 6 can be left empty
\cvitem{Téma práce:}{\emph{Kartografické vyjadřovací prostředky v prostředí Google Maps}}

\cventry{2009}{ukončeno}{Masarykova Univerzita}{Brno}{Mediální studia a žurnalistika}{}  % arguments 3 to 6 can be left empty

\cventry{2001-2009}{ukončeno maturitou}{Gymnázium Hejčín}{Olomouc}{}{}  % arguments 3 to 6 can be left empty


\section{Schopnosti}
\cvitemwithcomment{Programování}{PHP, Python}{vývoj v Nette framework, automatizace GIS procesů v Pythonu}
\cvitemwithcomment{Skriptování}{Javascript}{mapové knihovny: Leaflet, OpenLayers, Google Maps}
\cvitemwithcomment{GIS}{QGIS, OpenJump, GDAL, ArcMap, Geoserver, PostGIS}{pokročilý uživatel}
\cvitemwithcomment{IT}{Různé}{Linux, Git, \LaTeX, Inkscape, GIMP}
\cvitemwithcomment{Jazyk}{Angličtina}{pokročilý, čtení dokumentace, přispívání do fór a mailing listů}

\section{Zájmy}
\cvdoubleitem{Sport}{horská turistika, beach volejbal, cyklistika, plavání, via ferrata }{Literatura}{četba, psaní}
\cvdoubleitem{Technologie}{open source}{}{}

\section{Ostatní}
Jsem vytrvalý. V roce 2013 jsem zdolal trať Beskydské sedmičky (85~km) za 23h\,9min\,54s.\\
Jsem pečlivý. Úkoly plním včas. Řídím se heslem \emph{co můžeš odložit na zítra, udělej dnes, abys zítra mohl dělat to, co opravdu chceš}.\\
Jsem zvídavý. Zajímá mě, jak co funguje. Když to nevím, snažím se na to přijít.\\
Mám cit pro detail. Vadí mi, když věci nejsou perfektní. Občas k tomu přitom chybí tak málo.\\
\end{document}